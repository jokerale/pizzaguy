\documentclass[10pt]{article}

\usepackage[
    style=numeric,%alphabetic, 
    backend=bibtex,
    sorting=none
    ]{biblatex}
\bibliography{refs}{}


\title{Pizza Guy}
\date{\today}
\author{Giacomo Boldini, Alessio Diana, Federica Zaglio}

\begin{document}

	\maketitle

	\begin{abstract}

		In this report, we will describe the Pizza Guy project. This project 
		aims to find the best possible schedule of a set of orders in order to
		minimize the distance travelled by the deliverers in a evening of 
		work.\\	In the section \textbf{xyz} we will describe the project, in 
		\textbf{xyz} we will the describe the model and the implementation 
		including some euristichs and symmetry breaking	strategies. In the
		section \textbf{xyz} we will describe the results and the possible 
		search strategies.
		
	\end{abstract}

	\tableofcontents

	\section{Assignement}
	
	The assignement gives us the following information:
	\begin{enumerate}
		\item An order is made of:
		\begin{enumerate}
			\item the address of the delivery
			\item the desired time
			\item the number of pizzas
		\end{enumerate}
		\item The times of the delivery have a granularity of 15 minutes (from 
			19.00 to 22.00)
		\item The delivery window is up to 30 minutes later than the delivery 
			time
		\item We know the travel time between the nodes
		\item Every deliverer can carry at most 16 pizzas
		\item Multiple travel from and to the pizzeria are allowed even in the 
			delivery window
	\end{enumerate}

	\subsection{Goal}

	Our goal is:
	\begin{enumerate}
		\item Write a program that solves the problem
		\item Write a benchmark suite using 5 different town of increasing size
			and 10 input sets for each one. The input sets must enforce that all
			the deliverers do at least two travels
		\item On one configuration, one that runs in a couple of minutes, try
			different search strategies. Use the most promising one to solve a
			difficult input set
	\end{enumerate}

	\section{Model}

	file: \texttt{an-idea-2.mzn}\\~\
	
	\subsection{Input data}

	\paragraph*{}
	The input consists of:
	\begin{enumerate}
		\item $d$: the number of deliverer available
		\item $mdist$: the 2-d matrix representing the travel distance between
			two nodes in the graph generated using Dijkstra between all the nodes 
			in the graph(including the ones that are not a destination)
		\item $k$: the side dimension of $mdist$
		\item a set of $N$ orders specifing the destination $dest$, the requested 
			time $orario$ and the number of pizzas $num\_pizzas$
	\end{enumerate}

	\subsection{Assumption}
	In order to make the problem a little easier, we made the following 
	semplification:
	\begin{enumerate}
		\item The graph is undirected so \texttt{mdist}	is symmetric and 
			\texttt{mdist[x, y] = mdist[y, x]}.
		\item A deliverer can do only a single travel in one half-an-hour. In other
			words, the orders that are assigned to him/her are delivered creating 
			a circuit starting from the pizzeria, visiting all the destinations 
			and ending in the pizzeria.
	\end{enumerate}

    \paragraph*{}
	The time is divided in 6 slots of half-an-hours (19.00 - 22.00). Let's call
	$h$ one of these slots of time.
	There are $d$ deliverers available for doing deliveries.
	Each deliverer $d$ has a set of orders to handle for each slot of time $h$.
	These informations are stored in a boolean 3-d matrix called 
	\texttt{scheduling}, in which: \texttt{scheduling[$d$, orderID, $h$] = true}
	means that the deliverer $d$ has to deliver the order with id 
	\texttt{orderID} in the time slot $h$. In every slot $h$, the deliverer must 
	reach all nodes specified as \textit{destinations} of the orders assigned to 
	him/her, starting from node $1$ (pizzeria) and returning back to it (a sort
	of "circuit" starting and ending with the same node, that reaches all the 
	nodes specified in the matrix).
	Obviously, one order can be delivered only once during the evening by only
	one deliverer, in the half-an-hour specified by the client.

	\paragraph*{}
	During the evening there are $N$ order to deliver, each of which is made of
	16 pizzas at most. This ensure that we need only a deliverer to deliver the 
	entire order.
	Since we can deliver more than one order in a single travel, we need to
	record the total number of pizzas that a deliverer has in his/her bag.\\
	To do this we use a 2-d matrix called \texttt{pizzas\_carried}, in which 
	the first dimension represents the deliverer(domain $[1,d]$) and the second 
	represents the half-an-hour considered(domain $[1,h]$).

	\paragraph*{}




	This first semplification doesn't allow a deliverer to do more than one
	"circuit" in a single time slot $h$. 




	Only total travelled distance is minimized (for now). Maybe a balancing 
	function is needed or maybe not.

\end{document}



