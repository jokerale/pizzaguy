\documentclass[10pt]{article}

\usepackage[
    style=numeric,%alphabetic, 
    backend=bibtex,
    sorting=none
    ]{biblatex}
\bibliography{refs}{}


\title{Pizza Guy}
\date{\today}
\author{Giacomo Boldini, Alessio Diana, Federica Zaglio}

\begin{document}

	\maketitle

	\begin{abstract}

		In this report, we will describe the Pizza Guy project. This project 
		aims to find the best possible schedule of a set of orders in order to
		minimize the distance travelled by the deliverers in a evening of 
		work. Our solution is developed using a CLP(FD) model and implemented 
		in MiniZinc. All the remaing part of the project(data retrieving, cleaning
		and visualization) are done using Python.\\	
		In the section \ref{Assignment} we will describe the problem and the goal
		we want to reach, in the section \ref{Model} we will the describe the 
		model and the implementation, including some euristichs and symmetry breaking
		strategies. In the section \ref{Results} we will describe the results and
		the possible search strategies. In the last section \ref{Future work} we 
		will propose some possible changes and improvements of the current model.
		
	\end{abstract}

	\tableofcontents

	\section{Assignment}
	\label{Assignment}
	
	The assignment gives us the following informations:
	\begin{enumerate}
		\item An order is made of:
		\begin{enumerate}
			\item the delivery address
			\item the desired delivery time
			\item the number of pizzas
		\end{enumerate}
		\item The delivery time has a granularity of 15 minutes (from 
			19.00 to 22.00)
		\item The delivery window is up to 30 minutes later than the delivery 
			time
		\item We know the travel time between nodes
		\item Every deliverer can carry at most 16 pizzas
		\item Multiple travel from and to the pizzeria are allowed even in the 
			delivery window
	\end{enumerate}

	\subsection{Goal}

	Our goal is:
	\begin{enumerate}
		\item Write a program that solves the problem (in MiniZinc)
		\item Write a benchmark suite using 5 different towns of increasing size
			and 10 input sets for each one. The input sets must enforce that all
			the deliverers do at least two travels
		\item On one configuration that runs in a couple of minutes, try
			different search strategies. Use the most promising one to solve a
			difficult input set
	\end{enumerate}

	\section{Model}
	\label{Model}

	%file: \texttt{an-idea-2.mzn}\\~\
	
	\subsection{Input data}

	\paragraph*{}
	The input consists in having:
	\begin{enumerate}
		\item $d$: the number of deliverers available
		\item $mdist$: the 2-d matrix representing the travel distance between
			every node in the graph generated using Dijkstra(including those
			that are not a destination)
		\item $k$: the side dimension of $mdist$
		\item a set of $N$ orders specifing the destination $dest$, the delivery 
			time $orario$ and the number of pizzas $num\_pizzas$
			%TODO: cambiare il nome della struttura dati "ORARIO"
	\end{enumerate}

	\subsection{Assumption}
	In order to make the problem a little easier, we made the following 
	semplification:
	\begin{enumerate}
		\item The graph is undirected so \texttt{mdist}	is symmetric and 
			\texttt{mdist[x, y] = mdist[y, x]}.
		\item A deliverer can do only a single travel in one half-an-hour. In other
			words, the orders that are assigned to him/her are delivered creating 
			a circuit starting from the pizzeria, visiting all the destinations 
			and ending in the pizzeria.
	\end{enumerate}

	\subsection{Implementation}
	
	The time is divided in 6 slots of half-an-hours (19.00 - 22.00). Let's call
	$h$ one of these slots of time.
	There are $d$ deliverers available for doing deliveries.
	Each deliverer $d$ has a set of orders to handle, for each slot of time $h$.

    \paragraph*{Scheduling}
	These informations are stored in a boolean 3-d matrix called 
	\texttt{scheduling}, in which: \texttt{scheduling[$d$, orderID, $h$] = true}
	means that the deliverer $d$ has to deliver the order with id 
	\texttt{orderID} in the time slot $h$. In every slot $h$, the deliverer must 
	reach all nodes specified as \textit{destinations} of the orders assigned to 
	him/her, starting from node $1$ (pizzeria) and returning back to it (a sort
	of "circuit" starting and ending with the same node, that reaches all the 
	nodes specified in a matrix row).

	Obviously, one order can be delivered only once during the evening by only
	one deliverer, in the half-an-hour specified by the client.

	\paragraph*{Pizzas\_carried}
	During the evening there are $N$ orders to be delivered, each of which is made of
	16 pizzas at most. This ensure that we need only a deliverer to deliver the 
	entire order.
	Since we can deliver more than one order in a single travel, we need to
	record the total number of pizzas that a deliverer has in his/her bag.\\
	To do this, we use a 2-d matrix called \texttt{pizzas\_carried}, in which 
	the first dimension represents the deliverer(domain $[1,d]$) and the second 
	represents the half-an-hour considered(domain $[1,h]$).

	\paragraph*{Ea}
	In order to ensure that all deliveries arrive in time, we use \texttt{ea} which
	stands for estimated arrival. This array contains, for every order, the arrival 
	time, in minutes, from the starting time(19.00). At every order, we sum the number 
	of minutes that must pass before the delivery take place, so doing this way,
	we are sure that the delivery is done in the right half-an-hour.
	For example, if the delivery is requested for 20.00, the destination is five minutes 
	far from the pizzeria and no other delivery is made before that, we will sum 60 to 
	be sure about the right half-an-hour and 5 for the travel time. So the \texttt{ea}
	for this delivery will be 65.

	\paragraph*{P} The main goal of the project is to minimize the travel distance 
	covered by all the deliverers in the evening. To do so, a possible strategy
	is to deliver more than one delivery in a single travel. This is convenient 
	if the deliveries are made in a way that minimize the travel between the nodes.
	To encode this information, we use \texttt{P}, which is a 3-d matrix that stores
	all the nodes that every deliverer has to make in one half-an-hour. The useful thing
	about this matrix is that it allows us to control the order of each delivery made 
	by every deliverer in one half-an-hour and compute the travel distance.
	
	\paragraph*{X} This 3-d matrix is used to support the creation of all the other 
	constraints because is used to make easier the computation of the distancies.
	% magari un esempio non sarebbe male

	\paragraph*{Distances}
	Using the data structures seen before, we use this 3-d matrix to store all
	the partial distances needed to reach every delivery node. Using this matrix we can 
	enforce that all the deliveries are delivered in time and, since we want to minimize 
	the travel distance, it allows us to combine deliveries that belongs to different
	time slots(eg. 19.15 and 19.30) but only if they are adjacent because of the maximum 
	delivery time of 30 minutes.

	\paragraph*{}


	

	\section{Results}
	\label{Results}

	\section{Future work}
	\label{Future work}
	Only total travelled distance is minimized (for now). Maybe a balancing 
	function is needed or maybe not.

\end{document}



