\documentclass[10pt]{article}

\usepackage[
    style=numeric,%alphabetic, 
    backend=bibtex,
    sorting=none
    ]{biblatex}
\bibliography{refs}{}


\title{Report}
\date{\today}
\author{We}

\begin{document}

	\maketitle

	\begin{abstract}

		AAAAA
		
	\end{abstract}

	\tableofcontents

	\section{Intro}

	Test 1.

	\subsection{Intro sub}

	\section{Model v0.1}

	file: \texttt{an-idea-2.mzn}\\~\

	The time is divided in 6 slots of half-an-hours (19.00 - 22.00). Let's call $h$ one of these slots of time. 

	There are $d$ deliverers available for doing deliveries.

	Each deliverer $d$ has a set of orders to handle for each slot $h$. These informations are stored in a boolean 3-d matrix \texttt{scheduling}, in which:
	\begin{center}
		\texttt{scheduling[$d$, orderID, $h$] = true}
	\end{center} 
	means that the deliver $d$ has to deliver order with id \texttt{orderID} in the time slot $h$. In every slot $h$, the deliver must reach all nodes specified as \textit{destinations} of the order assigned to him/her, starting from node $1$ (pizzeria) and return back to it (a sort of "circuit" starting and ending with same node, that reach (at least) a specific set of nodes). 

	This first semplification doesn't allow a deliverer to do more than one "circuit" in a single time slot $h$. 

	Only total travelled distance is minimized (for now). Maybe a balancing function is needed or maybe not.


\end{document}



